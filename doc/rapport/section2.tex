\section{Recherches}
\subsection{Condition d'existence}
Allen J. Schwenk démontra en 1991 que trois conditions empèchent la réalisation d'un tour de cavalier sur un échiquier rectangulaire $m \times n$  pour tout $m \leq n$. Si aucune de ces condition n'es remplie alors au moins un circuit hamiltonien existe (par extension un chemin hamiltonien existe également).
\begin{enumerate}
\item $m$ et $n$ sont impaires
\item $m = 1,2,$ ou $4$
\item $m = 3$ et $n = 4,6$ ou $8$
\end{enumerate}
%démontrer les regles éventuellement

\subsection{Algorithmes de parcourt}
Le problème du cavalier étant connu et étudié depuis plusieurs siècles, un certain nombre d'algorithmes existent pour trouver un circuit ou un chemin hamiltonien.

\subsubsection{Force brute}
\subsubsection{Warnsdorf}
\subsubsection{Parberry}
\subsubsection{Réseaux neuronaux}