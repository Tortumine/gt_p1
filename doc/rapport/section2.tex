\section{Recherches}
\subsection{Condition d'existence}
Allen J. Schwenk démontra en 1991 que trois conditions empêchent la réalisation d'un tour de cavalier sur un échiquier rectangulaire $m \times n$  pour tout $m \leq n$. Si aucune de ces condition n'est remplie alors au moins un circuit hamiltonien existe (par extension un chemin hamiltonien existe également).
\begin{enumerate}
\item $m$ et $n$ sont impaires
\item $m = 1,2,$ ou $4$
\item $m = 3$ et $n = 4,6$ ou $8$
\end{enumerate}

\subsection{Algorithmes de parcourt}
Le problème du cavalier étant connu et étudié depuis plusieurs siècles, un certain nombre d'algorithmes existent pour trouver un circuit ou un chemin hamiltonien.

\subsubsection{Force brute}
Cette méthode permet de trouver un chemin/circuit à coup sûr si un tel chemin existe. Le principal inconvénient de cette technique c'est qu'elle nécessite un long temps de calcul. Pour trouver un chemin/circuit, nous établissons un arbre de parcours. La racine de l'arbre est la case de départ. Les branches représentent les déplacements valides possibles vers les autres cases. L'algorithme doit parcourir tout l'arbre des possibilités. S’il tombe sur un cul de sac (c'est à dire si le sous-arbre parcouru est de hauteur inférieure au nombre de cases de l'échiquier), il revient en arrière. Tous les chemins possibles sont ainsi testés.

Trouver un seul chemin/circuit peut prendre quelque secondes ou minutes si l'échiquier est suffisemment petit. A ce moment là, l'algorithme de parcourt s'arrête et la solution est affichée. L'énoncé cependant implique de trouver tous les chemins ou circuits sur l'échiquier afin de les énumérer. Ceci augmente grandement les temps de calcul.  Pour un échiquier de taille $8 \times 8$, il y a $ 4 \times 10^{51}$ suites de mouvements possibles. Ce qui conduit à des temps de calculs trop importants pour les machines actuelles. 
Cette méthode est la seul connue pour énumérer tous les chemins et circuits.

L'encyclopédie en ligne des suites de nombres entiers (OEIS) comporte la liste du nombre de chemins hamiltoniens en fonction de la taille d’un échiquier carré. Cette suite nommée A165134 permet de comparer les résultats obtenus expérimentalement avec les résultats officiels afin de valider l’efficacité de l’algorithme.
\subsubsection{Warnsdorf}
La loi de Warnsdorff est une heuristique développée par H. C. von Warnsdorff en 1823 qui permet de trouver un chemin hamiltonien sur un échiquier carré. L'heuristique se base sur le graphe présenté en figure ~\ref{cavalier_graphe} où chaque nœud se voit attribuer un poids correspondant au nombre de nœuds accessibles. On démarre le chemin sur un nœud au hasard. De ce nœud, on se déplace vers le nœud de poids le plus faible. Si plusieurs nœuds de même poids sont accessibles, ceux-ci doivent être départagés. La méthode a l'avantage d'avoir une complexité linéaire en aire (n*n). 

La méthode de départage la plus simple est la sélection aléatoire du nœud suivant. Cette méthode peut paraître simpliste, mais elle est pourtant efficace jusqu'à une certaine taille d'échiquier.

De nombreux mathématiciens se sont basés sur la loi de Warnsdorff pour y incorporer leur propre méthode de départage. Nous nous sommes plus particulièrement intéressés aux travaux de D. Squirrel, qui a élaboré en 1996 une méthode pour départager les candidats. Celle-ci permet de trouver à coup sûr un chemin hamiltonien pour toute taille d'échiquier carré, si tant est que celui-ci existe. Pour ce faire, toutes les cases accessibles depuis une case sont numérotées de 1 à 8 dans le sens horlogique, la case 1 étant celle en haut à gauche et la 8 en haut à droite. Selon la taille de l'échiquier et la position du cavalier, un ordre de parcours est définit. Par exemple : 34261578. Le chiffre à gauche est la première case à visiter et celle à droite est la dernière. Lorsque plusieurs cases ont le même poids, elles sont départagées par cet ordre de parcours. Nous avons retranscrit l'algorithme proposé par Squirrel dans un programme en C. Nous avons pu observer la fiabilité de ses résultats. Sa méthode a également une complexité O(n*n).
\subsubsection{Echiquier rectangulaire}
La forme de l'échiquier influence évidemment les méthodes de recherches du problème du cavalier. Il est à noter que la loi de Warnsdorff avec départage aléatoire peut fonctionner sur des échiquiers rectangulaires, tel que nous l'avons testé. Néanmoins, l'heuristique présente évidemment les mêmes faiblesses que pour les échiquiers carrés.

Nous somme attardés sur les recherches de Shun-Shii Lin et Chung-Liang Wei, qui ont réussi à créer un algorithme pour trouver à coup sur un chemin ou un circuit Hamiltonien sur un échiquier rectangulaire de taille m*n.  A condition qu'un tel chemin existe. Ils se sont basés sur les recherches de Parberry, qui a développé un algorithme basé sur le principe du "diviser pour régner". Avec ce principe, l'échiquier est divisé en petits échiquiers. Ensuite, un chemin hamiltonien est trouvé pour chaque petit échiquier. Ensuite, les chemins sont reliés ensemble pour former un seul grand échiquier. Par manque de temps, nous n'avons implémenté qu'une partie de l'algorithme, permettant de résoudre le problème avec un échiquier de taille 3*m. Encore une fois avec une complexité O(n*m).

