\section{Annexe}
\subsection{Compilation}
La compilation se fait en utilisant cmake et make.

Depuis le répertoire source:
\begin{itemize}
\item mkdir build
\item cd build
\item cmake ..
\item make
\end{itemize}
\subsection{Execution}
Deux exécutables sont généres, \textbf{path} et \textbf{path\_all}

\paragraph{path}Il n'y a pas d'arguments pour path. Tout se fait via un menu.
\paragraph{path\_all} peut predre jusqu'à 3 arguments. Si ils sont mal passés l'application utilisera les arguments par défaut -o 5 5.
\begin{itemize}
\item Type de parcourt ouvert -o ou fermé -c
\item Largeur de la grille u\_int
\item Hauteur de la grille u\_int
\end{itemize}
\subsection{Sources}
\begin{itemize}
  \item \textbf{Wikipedia: Problème du cavalier} https://fr.wikipedia.org/wiki/Probl%C3%A8me_du_cavalier
    \item \textbf{Wikipedia: Problème du cavalier} \url{https://fr.wikipedia.org/wiki/Probl%C3%A8me_du_cavalier}
	\item \textbf{Zanotti: Le problème du cavalier} \url{http://zanotti.univ-tln.fr/ALGO/I51/Cavalier.html}
	\item \textbf{Baylede: Problème du cavalier} \url{http://bayledes.free.fr/carres_magiques/Cavaliers.html}
	\item \textbf{Graphs and Graph Algorithms} \url{http://interactivepython.org/runestone/static/pythonds/Graphs/toctree.html}
	\item \textbf{Knight’s Tour Analysis} \url{http://interactivepython.org/runestone/static/pythonds/Graphs/KnightsTourAnalysis.html}
	\item \textbf{GeeksForGeeks: Warnsdorff} \url{http://www.geeksforgeeks.org/warnsdorffs-algorithm-knights-tour-problem/}
	\item \textbf{A Warnsdorff-Rulle Algorithms for Knight’s Tours on Square Chessbords}\textit{Squirrell et Cull 1996} \url{http://math.oregonstate.edu/~math_reu/proceedings/REU_Proceedings/Proceedings1996/1996Squirrel.pdf}
	\item \textbf{Zestedesavoir: Graphes et représentation de graphe} \url{https://zestedesavoir.com/tutoriels/681/a-la-decouverte-des-algorithmes-de-graphe/727_bases-de-la-theorie-des-graphes/3352_graphes-et-representation-de-graphe/}
	\item \textbf{Optimal algorithms for constructing knight's tours on arbitrary n*m chessboards}\textit{Shun-Shii Lin et Chung-Liang Wei 2005} \url{http://www.sciencedirect.com/science/article/pii/S0166218X04003488}

\end{itemize}
